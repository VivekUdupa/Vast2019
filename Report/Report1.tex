%\documentclass[12pt]{article}
\documentclass[12pt]{extarticle}
\usepackage[utf8]{inputenc}
\usepackage[margin=0.5in]{geometry}
\usepackage{enumitem}
\usepackage{graphicx}
\usepackage{float}
\usepackage{romannum}
 
 % Title page
\title{VAST 2019 Weekly Report 1}
\author{Vivek Koodli Udupa}
\date{\today}

\begin{document}
\pagenumbering{arabic}
\maketitle

% Introduction
\begin{centering}
	\section{Introduction}
\end{centering} \noindent
This report considers the VAST 2019 Mini Challenge 1 (MC1), where visual analytics is implemented to help a fictitious city grapple with the aftermath of an earthquake that damages their nuclear power plant. \\

St. Himark is a beautiful community located at the Ocenaus sea. It is a small community with almost everything it needs to sustain a spirited civilization. St. Himark is primarily powered by the Always Safe Nuclear Power Plant. This was true until the disaster struck. Now, Mayor Jordan, city officials, and emergency services are overwhelmed and are desperate for assistance in understanding the true situation on the ground and how best to deploy the limited resources available to this relatively small community. \\

In a prescient move of community engagement, the city had released a new damage reporting mobile application, 'RUMBLE', that allows citizens to report damages that they see in their neighborhood. The challenge is to use app responses in conjunction with shake maps of the earthquake strength to identify areas of concern and advise emergency planners respond to damages more efficiently. \\

RUMBLE stores the Time-stamp and location ID for each entry. The time stamp is is YYYY-MM-DD hh:mm:ss format. The location ID can be any integer in the range 1 to 19, representing different cities in the community. The damages to Sewer \& Water, Power, Roads \& Bridges, Medical, Buildings are reported on a scale of 0 to 10, 10 being the highest. The shake intensity is also represented on a identical scale. \\

The data for MC1 is included in the 'mc1-reports-data.csv' CSV file that spans over the entire length of the event. It is consisted of categorical reports of shaking/damage to the neighborhood over time. \\

 
%All plots and description
\begin{centering}
	\section{Analysis and Visualization}
\end{centering}
The tasks of the MC1 are split into two parts: \\

\begin{enumerate}[itemsep=0mm]
	\item How to prioritize neighborhood for response?
	\item Which parts of the city have taken the most damage?
\end{enumerate}
\noindent
As described in the Introduction, the community is encoded into location ID's that range from 1 to 19.  \\
Figure \ref{fig:map} represents the map of St.Himark highlighting different cities along with their respective location codes. \\
Figure \ref{fig:shakemap} shows where the earthquake's epicenter originates as well as the intensity of shake felt across different locations. 

\begin{figure}[H]
	\centering
	\begin{minipage}{0.5\textwidth}
		\centering
		\includegraphics[width=\textwidth]{Images/map.png}
		\caption{St.Himark Neighborhood Map}
		\label{fig:map}
	\end{minipage}%
	\begin{minipage}{0.5\textwidth}
		\centering
		\includegraphics[width=\textwidth]{Images/shakemap.png}
		\caption{St.Himark Shake Map}
		\label{fig:shakemap}
	\end{minipage}
\end{figure} 

Based on preliminary visual comparison, it can be noted that the cities Old Town (location Id: 3), Safe Town (location Id: 4), Wilson Forest (location Id: 7) and Pepper Mill (location Id: 12) are the ones that are experiencing the maximum shake among all the cities. On the given scale, they are experiencing Moderate(\Romannum{4}) shake. 
	
\subsection{Prioritizing Neighborhood Response:}
During a Calamity, the foremost concern is to keep the casualties to a minimum. This means deploying immediate medical assistance to disaster struck areas. There are a total of 7 hospitals spread across St. Himark as shown in Figure \ref{fig:map}.  \\

Figure \ref{fig:medical} shows the damage hospitals sustained in different localities. As expected, the hospital at Old Town (location Id: 3) is the one with maximum reported damage. The emergency responders probably have to divert all possible injured to Downtown (location Id: 6) which has sustained about half the medical building damage as compared to Old Town.  \\

Transporting injured to different locations comes at the cost of Time. It is only possible to transport the minorly injured to farther locations. This transportation can work only if the roads and bridges are intact. Let us take a look at the damages the Roads have taken at different location.  \\

Figure \ref{fig:road} represents the average reported damage roads and bridges sustained during the earthquake. It is to be noted that the graph peaks at Location 8, Scenic Vista, which was safe from the earthquake according to the shake map. However, Old town has sustained an average reported damage of 7.26 out of 10, which is quiet a lot. This could cause major delays in transporting people out of Old Town. Relocating only the non-fatal injuries might be a good idea. Safe Town (location 4) is another major concern. It does not have its own hospital and the average road damage is reported at 4.21. This is not the worst, but certain roads might be inaccessible. Proper communication between the emergency response team can reduce the transportation time. The nearest hospitals to Safe Town are located at Terrapin Springs (location 11), Broadview (location 9) and Southton (location 16). \\ 


\begin{figure}[H]
\centering
	\includegraphics[width=\linewidth]{Images/medical.png}
	\caption{Location Vs Reported Medical Building Damage }
	\label{fig:medical}
\end{figure}

\begin{figure}[H]
\centering
	\includegraphics[width=\linewidth]{Images/Road.png}
	\caption{Location Vs Reported Roads and Bridges Damage }
	\label{fig:road}
\end{figure}
 
 \newpage
\begin{centering}
	\section{Conclusion}
\end{centering}



\end{document}


